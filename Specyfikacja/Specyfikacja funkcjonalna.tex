\documentclass[]{article}
\usepackage{lmodern}
\usepackage{amssymb,amsmath}
\usepackage{ifxetex,ifluatex}
\ifnum 0\ifxetex 1\fi\ifluatex 1\fi=0 % if pdftex
  \usepackage[T1]{fontenc}
  \usepackage[utf8]{inputenc}
%\setcounter{secnumdepth}{0}
 \usepackage[table,xcdraw]{xcolor}
 \usepackage[margin=1.5in]{geometry}
 \usepackage[T1]{fontenc}
\usepackage[tableposition=top]{caption}
\usepackage{tabularx}
\usepackage{xcolor}
\usepackage{hyperref}
\hypersetup{
    colorlinks=true,
    linkcolor=blue,
    filecolor=magenta,      
    urlcolor=cyan,
}



\title{Specyfikacja funkcjonalna projektu zespołowego \textbf{AiSD GR2}}
\author{Martyna Jakubowska, Hubert Nakielski, Artur Prasuła}
\date{Grudzień 2020}


\begin{document}
\maketitle


\section{Opis ogólny}
\subsection{Nazwa programu} % Hubert
\subsection{Poruszany problem} % Martyna
\subsection{Użytkownik docelowy}
Program dedykowany jest służbie zdrowia do planowania najbardziej optymalnych tras karetek.
Użytkownikami programu będą dyspozytorzy, którzy wydają rozkazy co do tras karetek.


\section{Opis funkcjonalności}
\subsection{Jak korzystać z programu} % Martyna
\subsection{Możliwości programu} % Artur


\section{Format danych i struktura plików}
\subsection{Struktura katalogów} % Hubert
\subsection{Przechowywanie danych w programie} % Hubert
\subsection{Dane wejściowe} % Artur
\subsection{Dane wyjściowe} % Martyna


\section{Scenariusz działania programu}
\subsection{Scenariusz ogólny} % Hubert
\subsection{Scenariusz szczegółowy} % Wszyscy
\subsection{Ekrany działania programu} % Martyna


\section{Testowanie}
Poszczególne klasy programu zostaną przetestowane za pomocą testów jednostkowych. Do tego posłuży nam narzędzie JUnit. Współpraca klas w programie oraz działanie graficznego interfejsu użytkownika zostanie przetestowane przez nas ręcznie w trakcie implementacji.

\end{document}
