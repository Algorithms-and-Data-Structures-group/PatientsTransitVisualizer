\documentclass[]{article}
\usepackage{ifxetex,ifluatex}
\ifnum 0\ifxetex 1\fi\ifluatex 1\fi=0 % if pdftex
\usepackage[T1]{fontenc}
\usepackage[utf8]{inputenc}
%\setcounter{secnumdepth}{0}
\usepackage[table,xcdraw]{xcolor}
\usepackage[margin=1.5in]{geometry}
\usepackage[tableposition=top]{caption}
\usepackage{tabularx}
\usepackage{hyperref}
\hypersetup{
colorlinks=true,
linkcolor=blue,
filecolor=magenta,
urlcolor=cyan,
}



\title{Specyfikacja implementacyjna projektu zespołowego \textbf{AiSD GR2}}
\author{Martyna Jakubowska, Hubert Nakielski, Artur Prasuła}
\date{Grudzień 2020}


\begin{document}
    \maketitle
    
    \section{Informacje Ogólne}
    Program można uruchomić za pomocą naciśnięcia myszką na ikonę programu.
    Dlatego też, program nie przyjmuje żadnych parametrów uruchomieniowych.
    Okno programu po włączeniu będzie mieć rozmiar 800x600 pikseli.
    Dzięki temu, program zmieści się niemalże na każdym sprzęcie.
    Dodatkowo rozmiar okna, będzie mógł zostać zmieniony ręcznie przez użytkownika.
    Program pojawi się na ekranie tak, aby środek okna programu był środkiem monitora użytkownika.
    Wyłączenie programu będzie możliwe za pomocą przycisku w prawym górnym rogu programu.
    
    
    \section{Opis pakietów}
        \subsection{ptv}
        \subsection{ptv.module}
        \subsection{ptv.module.path} % Hubert
        \subsection{ptv.module.data} % Hubert
        \subsection{ptv.module.borders} % Martyna
		
        \subsection{ptv.module.reader}
            Pakiet \textbf{ptv.module.reader} odpowiada za wczytywanie danych z plików tekstowych.
            Znajdziemy w nim dwie klasy: \textbf{MapFileReader}, \textbf{PatientsFileReader}.
            Pakiet ten korzysta z pakietu \textbf{ptv.module.data}, ponieważ tam zawarte są klasy wczytywanych z pliku obiektów.
        
            \subsubsection{MapFileReader}
                Klasa \textbf{MapFileReader} służy do wczytywania szpitali, obiektów i dróg z pliku tekstowego.
                
                \paragraph{Pola}
                    \begin{itemize}
                        \item \textbf{private int lineNumber} - numer aktualnie analizowanej linii
                        \item \textbf{private Hospital[] loadedHospitalsById} - tablica zawierająca wczytane szpitale, dany indeks w tablicy to szpital o tym id
                        \item \textbf{private Map<String, Hospital> loadedHospitalsByName} - HashMap'a zawierająca wczytane szpitale, dany klucz to nazwa szpitala kryjącego się pod tym kluczem
                        \item \textbf{private Facility[] loadedFacilitiesById} - tablica zawierająca wczytane obiekty, dany indeks w tablicy to obiekt o tym id
                        \item \textbf{private Map<String, Facility> loadedFacilitiesByName} - HashMap'a zawierająca wczytane szpitale, dany klucz to nazwa szpitala kryjącego się pod tym kluczem
                        \item \textbf{private Distance[] loadedDistancesById} - tablica zawierająca wczytane połączenia szpitali, dany indeks w tablicy to połączenie o tym id
                    \end{itemize}
                
                \paragraph{Metody}
                    \begin{itemize}
                        \item \textbf{public Map readFile(String filePath)}\\
                            Metoda służy do wczytania mapy z pliku tekstowego do którego prowadzi ścieżka \textbf{filePath}.\\
                            Metoda zwraca obiekt klasy Map, który zawiera dane wczytane z pliku.\\
                            Metoda wczytuje linia po linii i wywołuje odpowiednia metodę w zależności od aktualnie wczytywanego typu obiektów: readHospital(), readFacility(), readDistance().\\
                            Metoda wyrzuca wyjątek IllegalArgumentException jeśli,
                                \textbf{filePath} jest null'em,
                                nie może wczytać z danego pliku,
                                plik ma niepoprawny format danych.
                                
                        \item \textbf{private void readHospital(String[] line, Map map}\\
                            Metoda służy do dodania szpitala do \textbf{map} z podzielonej linii \textbf{line}.
                            Metoda sprawdza czy dana linia ma odpowiedni format i wywołuje \textbf{canLoadHospital()}.
                            Jeśli linia jest poprawna i można wczytać ten szpital, metoda dodaje go do \textbf{map}.\\
                            Metoda wyrzuca wyjątek IllegalArgumentException jeśli,
                                \textbf{line} ma niepoprawny format danych,
                                szpital o takim id lub nazwie już został wcześniej wczytany.
                        
                        \item \textbf{private void readFacility(String[] line, Map map}\\
                            Metoda służy do dodania obiektu do \textbf{map} z podzielonej linii \textbf{line}.
                            Metoda sprawdza czy dana linia ma odpowiedni format i wywołuje \textbf{canLoadFacility()}.
                            Jeśli linia jest poprawna i można wczytać ten obiekt, metoda dodaje go do \textbf{map}.\\
                            Metoda wyrzuca wyjątek IllegalArgumentException jeśli,
                                \textbf{line} ma niepoprawny format danych,
                                obiekt o takim id lub nazwie już został wcześniej wczytany.
                                
                        \item \textbf{private void readDistance(String[] line, Map map}\\
                            Metoda służy do dodania drogi do \textbf{map} z podzielonej linii \textbf{line}.
                            Metoda sprawdza czy dana linia ma odpowiedni format i wywołuje \textbf{canLoadDistance()}.
                            Jeśli linia jest poprawna i można wczytać tę drogę, metoda dodaje ją do \textbf{map}.\\
                            Metoda wyrzuca wyjątek IllegalArgumentException jeśli,
                                \textbf{line} ma niepoprawny format danych,
                                droga o takim id już została wcześniej wczytana.
                                
                        \item \textbf{private boolean canLoadHospital(int id, String name)}\\
                            Metoda służy do sprawdzenia czy szpital o takim id i nazwie został już wcześniej wczytany.\\
                            Metoda zwraca \textbf{true} jeśli taki szpital nie został wcześniej wczytany,
                            a \textbf{false} w przeciwnym wypadku.\\
                            Sprawdza to za pomocą pól \textbf{loadedHospitalsById, loadedHospitalsByName}.
                            
                        \item \textbf{private boolean canLoadFacility(int id, String name)}\\
                            Metoda służy do sprawdzenia czy obiekt o takim id i nazwie został już wcześniej wczytany.\\
                            Metoda zwraca \textbf{true} jeśli taki obiekt nie został wcześniej wczytany,
                            a \textbf{false} w przeciwnym wypadku.\\
                            Sprawdza to za pomocą pól \textbf{loadedFacilitiesById, loadedFacilitiesByName}.
                            
                        \item \textbf{private boolean canLoadDistance(int id)}\\
                            Metoda służy do sprawdzenia czy droga o takim id została już wcześniej wczytana.\\
                            Metoda zwraca \textbf{true} jeśli taka droga nie została wcześniej wczytana,
                            a \textbf{false} w przeciwnym wypadku.\\
                            Sprawdza do za pomocą pola \textbf{loadedDistancesById}.
                    \end{itemize}
                    
            \subsubsection{PatientsFileReader}
                Klasa \textbf{PatientsFileReader} służy do wczytania listy pacjentów z pliku tekstowego.
                
                \paragraph{Pola}
                    \begin{itemize}
                        \item \textbf{private int lineNumber} - numer aktualnie analizowanej linii
                        \item \textbf{private Map<Integer, Patient> loadedPatients} - HashMap'a zawierająca wczytanych pacjentów, klucz jest to id danego pacjenta
                    \end{itemize}
                
                \paragraph{Metody}
                    \begin{itemize}
                        \item \textbf{public List<Patient> readFile(String filePath)}
                            Metoda służy do wczytania listy pacjentów z pliku tekstowego do któego prowadzi ścieżka \textbf{filePath}.\\
                            Metoda zwraca obiekt klasy List<>, który zawiera listę pacjentów wczytanych z pliku.\\
                            Metoda wczytuje linia po linii i wywołuje metodę readLine().\\
                            Metoda wyrzuca wyjątek IllegalArgumentException jeśli,
                                \textbf{filePath} jest null'em,
                                nie może wczytać z danego pliku,
                                plik ma niepoprawny format danych.
                        
                        \item \textbf{private void readLine(String[] line, List<Patient> patients}\\
                            Metoda służy do dodania pacjenta do \textbf{patients} z podzielonej linii \textbf{line}.
                            Metoda sprawdza czy dana linia ma odpowiedni format i wywołuje \textbf{canLoadPatient()}.
                            Jeśli linia jest poprawna i można wczytać tego pacjenta, metoda dodaje ją do \textbf{patients}.\\
                            Metoda wyrzuca wyjątek IllegalArgumentException jeśli,
                                \textbf{line} ma niepoprawny format danych,
                                pacjent o takim id już został wcześniej wczytany.
                                
                        \item \textbf{private boolean canLoadPatient(int id)}\\
                            Metoda służy do sprawdzenia czy pacjent o takim id został już wcześniej wczytany.\\
                            Metoda zwraca \textbf{true} jeśli taki pacjent nie został wcześniej wczytany,
                            a \textbf{false} w przeciwnym wypadku.\\
                            Sprawdza to za pomocą pola \textbf{loadedPatients}.\\
                    \end{itemize}

        \subsection{ptv.module.simulation} % Artur
        \subsection{ptv.controllers} % Martyna
        \subsection{ptv.views} % Martyna
        \subsection{ptv.content}

    
    \section{Opis GUI} % Artur
    
    
    \section{Testowanie}
        \subsection{Użyte narzędzia}
        Głównym narzędziem, za pomocą którego program zostanie przetestowany, będzie \textbf{JUnit}.
        Posłuży on nam do napisania testów jednostkowych metod w klasach.
        Dzięki temu, będziemy mogli przetestować pojedyncze funkcjonalności programu.\\
        Testy integralności tych funkcjonalności i testy graficznego interfejsu, zostaną przetestowane przez programistów, w trakcie implementacji.
        
        \subsection{Konwencja}
        Testy danych klas za pomocą narzędzia \textbf{JUnit} zostaną napisane w odpowiadającej klasie i pakiecie
        w katalogu \textbf{test}.
        Metody testujące będą mieć nazwy odpowiadające temu, co metoda testuje.
        Użyta zostanie konwencja nazwnictwa metod testujących: \textbf{should + [co] + When + [warunek]}
        (np. shouldThrowIllegalArgumentExceptionWhenFilePathIsNull).\\
        Dzięki takiej konwencji rozmieszczenia plików testujących, łatwo znajdziemy testy danej klasy,
        a dzięki konwencji nazewnictwa metod, łatwo zrozumiemy co dana metoda testuje.
        
        \subsection{Warunki brzegowe}
        Wszystkie klasy z ważnymi i skomplikowanymi metodami, zostaną skrupulatnie przetestowane. Do tego zostaną użyte poprawne, jak i niepoprawne dane.
        \\
        Klasą, która jest najważniejsza jest klasa \textbf{PathFinder}. 
        Zostanie ona przetestowana dla danych zawierających różną liczbę połączeń pomiędzy szpitalami, a także gdy nie ma żądnych połączeń pomiędzy szpitalami.
        Testy będą się opierać głównie o to, by sprawdzić czy wyszukane połączenie, na pewno jest najkrótszym możliwym.
        Również zostanie wykonany test dla mapy, która nie zawiera żadnych szpitali.
        \\
        Klasy \textbf{MapFileReader} i \textbf{PatientsFileReader} są klasami najbardziej narażonymi na błędne dane.
        Zostaną one przetestowane, głównie dla danych niepoprawnych by sprawdzić
        czy algorytmy w nich zawarte potrafią znajdować błędy w pliku tekstowym.
        Te klasy zostaną przetestowane dla pustych plików oraz plików, które zawierają niepoprawne dane, takie jak:
        \begin{itemize}
            \item Liczby ujemne w miejscach gdzie są niedozwolone
            \item Powtarzające się id i nazwy (tylko w pliku z mapą)
            \item Połączenia pomiędzy nieistniejącymi szpitalami
            \item Dane, które nie są ani szpitalami, ani obiektami, ani drogami.
        \end{itemize}
        GUI zostanie przetestowane przez programistów ręcznie.
        Każda z kontrolek zostanie przetestowana pod względem poprawności działania.
        Sprawdzimy czy za pomocą suwaku szybkości animacji, można zatrzymać program.
        Sprawdzimy czy można uruchomić animację, gdy jest już ona uruchomiona.
        
    
    \section{Diagram klas} % Artur


\end{document}