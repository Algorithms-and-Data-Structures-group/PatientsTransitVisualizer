\documentclass[]{article}
\usepackage{ifxetex,ifluatex}
\ifnum 0\ifxetex 1\fi\ifluatex 1\fi=0 % if pdftex
\usepackage[T1]{fontenc}
\usepackage[utf8]{inputenc}
%\setcounter{secnumdepth}{0}
\usepackage[table,xcdraw]{xcolor}
\usepackage[margin=1.5in]{geometry}
\usepackage[tableposition=top]{caption}
\usepackage{tabularx}
\usepackage{hyperref}
\hypersetup{
colorlinks=true,
linkcolor=blue,
filecolor=magenta,
urlcolor=cyan,
}



\title{Specyfikacja implementacyjna projektu zespołowego \textbf{AiSD GR2}}
\author{Martyna Jakubowska, Hubert Nakielski, Artur Prasuła}
\date{Grudzień 2020}


\begin{document}
    \maketitle
    
    \section{Informacje Ogólne}
    Program można uruchomić za pomocą naciśnięcia myszką na ikonę programu.
    Dlatego też, program nie przyjmuje żadnych parametrów uruchomieniowych.
    Okno programu po włączeniu będzie mieć rozmiar 800x600 pikseli.
    Dzięki temu, program zmieści się niemalże na każdym sprzęcie.
    Dodatkowo rozmiar okna, będzie mógł zostać zmieniony ręcznie przez użytkownika.
    Program pojawi się na ekranie tak, aby środek okna programu był środkiem monitora użytkownika.
    Wyłączenie programu będzie możliwe za pomocą przycisku w prawym górnym rogu programu.
    
    
    \section{Opis pakietów}
        \subsection{ptv}
        \subsection{ptv.module}
        \subsection{ptv.module.path} % Hubert
        \subsection{ptv.module.data} % Hubert
        \subsection{ptv.module.borders} % Martyna
        \subsection{ptv.module.reader} % Artur
        \subsection{ptv.module.simulation} % Artur
        \subsection{ptv.controllers} % Martyna
        \subsection{ptv.views} % Martyna
        \subsection{ptv.content}

    
    \section{Opis GUI} % Artur
    
    
    \section{Testowanie}
        \subsection{Użyte narzędzia}
        Głównym narzędziem, za pomocą którego program zostanie przetestowany, będzie \textbf{JUnit}.
        Posłuży on nam do napisania testów jednostkowych metod w klasach.
        Dzięki temu, będziemy mogli przetestować pojedyncze funkcjonalności programu.\\
        Testy integralności tych funkcjonalności i testy graficznego interfejsu, zostaną przetestowane przez programistów, w trakcie implementacji.
        
        \subsection{Konwencja}
        Testy danych klas za pomocą narzędzia \textbf{JUnit} zostaną napisane w odpowiadającej klasie i pakiecie
        w katalogu \textbf{test}.
        Metody testujące będą mieć nazwy odpowiadające temu, co metoda testuje.
        Użyta zostanie konwencja nazwnictwa metod testujących: \textbf{should + [co] + When + [warunek]}
        (np. shouldThrowIllegalArgumentExceptionWhenFilePathIsNull).\\
        Dzięki takiej konwencji rozmieszczenia plików testujących, łatwo znajdziemy testy danej klasy,
        a dzięki konwencji nazewnictwa metod, łatwo zrozumiemy co dana metoda testuje.
        
        \subsection{Warunki brzegowe}
        Wszystkie klasy z ważnymi i skomplikowanymi metodami, zostaną skrupulatnie przetestowane. Do tego zostaną użyte poprawne, jak i niepoprawne dane.
        \\
        Klasą, która jest najważniejsza jest klasa \textbf{PathFinder}. 
        Zostanie ona przetestowana dla danych zawierających różną liczbę połączeń pomiędzy szpitalami, a także gdy nie ma żądnych połączeń pomiędzy szpitalami.
        Testy będą się opierać głównie o to, by sprawdzić czy wyszukane połączenie, na pewno jest najkrótszym możliwym.
        Również zostanie wykonany test dla mapy, która nie zawiera żadnych szpitali.
        \\
        Klasy \textbf{MapFileReader} i \textbf{PatientsFileReader} są klasami najbardziej narażonymi na błędne dane.
        Zostaną one przetestowane, głównie dla danych niepoprawnych by sprawdzić
        czy algorytmy w nich zawarte potrafią znajdować błędy w pliku tekstowym.
        Te klasy zostaną przetestowane dla pustych plików oraz plików, które zawierają niepoprawne dane, takie jak:
        \begin{itemize}
            \item Liczby ujemne w miejscach gdzie są niedozwolone
            \item Powtarzające się id i nazwy (tylko w pliku z mapą)
            \item Połączenia pomiędzy nieistniejącymi szpitalami
            \item Dane, które nie są ani szpitalami, ani obiektami, ani drogami.
        \end{itemize}
        GUI zostanie przetestowane przez programistów ręcznie.
        Każda z kontrolek zostanie przetestowana pod względem poprawności działania.
        Sprawdzimy czy za pomocą suwaku szybkości animacji, można zatrzymać program.
        Sprawdzimy czy można uruchomić animację, gdy jest już ona uruchomiona.
        
    
    \section{Diagram klas} % Artur


\end{document}